% In this file you should put the actual content of the blueprint.
% It will be used both by the web and the print version.
% It should *not* include the \begin{document}
%
% If you want to split the blueprint content into several files then
% the current file can be a simple sequence of \input. Otherwise It
% can start with a \section or \chapter for instance.

\section{Introduction}
We follow the paper ``On ordered semigroups'' by
N. G. Alimov.

\section{Content}

\begin{definition}\label{OrderedSemigroup}
An ordered semigroup $S$ is a semigroup with a linear order such that
for all $a,b,c\in S$ such that $a < b$
\begin{enumerate}
    \item $a + c < b + c$
    \item $c + a < c + b$
\end{enumerate}
\end{definition}

\begin{theorem}
An ordered semigroup is cancellative.
\end{theorem}
\begin{proof}
Let $S$ be an ordered semigroup. Let $a,b,c\in S$.

Assume that $a + c = b + c$.
Then if $a < b$ we have that $a + c < b + c$ which is a contradiction.
Likewise, if $b < a$, we have that $b + c < a + c$ which is a contradiction.
Therefore, $a = b$.

Assume that $c + a = c + b$.
Then if $a < b$ we have that $c + a < c + b$ which is a contradiction.
Likewise, if $b < a$, we have that $c + b < c + a$ which is a contradiction.
Therefore, $a = b$.
\end{proof}

\begin{definition}
An element $a$ of an ordered semigroup $S$ 
is positive if for all $x\in S$, $a+x > x$.
\end{definition}

\begin{definition}
An element $a$ of an ordered semigroup $S$
is negative if for all $x\in S$, $a+x < x$.
\end{definition}

\begin{definition}
An element $a$ of an ordered semigroup $S$
is zero if for all $x\in S$, $a+x = x$.
\end{definition}

\begin{theorem}
Each element $a$ of an ordered semigroup $S$ is either positive, negative, or zero.
\end{theorem}
\begin{proof}
Let $a\in S$ and $b\in S$. Since the $S$ is a linear order we have one of the following cases.

The first case is that $b + a > b$. Then for all $x\in S$ we have that $b + a + x > b + x$
and so $a + x > x$. Therefore, $a$ is positive.

The second case is that $b + a < b$. Then for all $x\in S$ we have that $b + a + x < b + x$
and so $a + x < x$. Therefore, $a$ is negative.

The last case is that $b + a = b$. Then for all $x\in S$ we have that $b + a + x = b + x$
and so $a + x = x$. Therefore, $a$ is zero.
\end{proof}

\begin{corollary}
Let $a$ be an element of an ordered semigroup $S$. 

If there exists a $b\in S$ such that $b + a > b$, $a$ is positive.

If there exists a $b\in S$ such that $b + a < b$, $a$ is negative.

If there exists a $b\in S$ such that $b + a = b$, $a$ is zero.
\end{corollary}

\begin{theorem}
Let $a$ and $b$ be elements of an ordered semigroup $S$.

If $a$ is negative and $b$ is positive, then $a < b$.
\end{theorem}
\begin{proof}
Let $a$ be negative and $b$ be positive. Then for all $x\in S$ we have that $a + x < x$.
Likewise, for all $x\in S$ we have that $b + x > x$.
Therefore, $a + x < x < b + x$ and so $a < b$.
\end{proof}

We endow our ordered semigroup with scalar multiplication by the positive natural numbers.

\begin{lemma}\label{DecreaseN}
Let $a$ and $b$ be elements of an ordered semigroup $S$.

For all $n > 1$, $n(a+b) = a + (n-1)(b+a) + b$.
\end{lemma}
\begin{proof}
Let $n=2$. Then $2(a+b) = a + (2-1)(b+a) + b$.

Assume that the statement holds for $n$.
Then we have that $(n+1)(a+b) = a + b + n(a+b) = a + b + a + (n-1)(b+a) + b = a + n(b+a) + b$.
\end{proof}

\begin{lemma}
Let $a$ and $b$ be elements of an ordered semigroup $S$.

If $a + b < b + a$, then for any $n\in \mathbb{N}^+$, we have that
\[na + nb < n(a+b) < n(b+a) < nb + na\]
\end{lemma}
\begin{proof}
If $n=1$, then we are done.

Assume that the statement holds for $n$.
Then we have that
\begin{align}
(n+1)a + (n+1)b &= a + na + nb + b \\
\text{by the induction hypothesis}\\
&< a + n(a + b) + b \\
\text{since $a+b < b+a$}\\
&< a + n(b + a) + b \\
\text{by the previous lemma}\\
&= (n+1)(a + b) \\
&< (n+1)(b + a) \\
\text{by the previous lemma}\\
&= b + n(a + b) + a \\
&< b + n(b + a) + a \\
\text{by the induction hypothesis}\\
&< b + nb + na + a \\
&= (n+1)b + (n+1)a
\end{align}
\end{proof}

\begin{definition}
Let $a$ and $b$ be elements of an ordered semigroup $S$ that are not zero.

Then $a$ is said to be Archimedean with respect to $b$
if there exists an $N\in \mathbb{N}^+$ such that for $n > N$,
the inequality $b < na$ holds if $b$ is positive,
and the inequality $b > na$ holds if $b$ is negative.
\end{definition}

\begin{definition}
An ordered semigroup is Archimedean if any two of its elements
of the same sign are mutually Archimedean.
\end{definition}

\begin{definition}
Let $a$ and $b$ be elements of an ordered semigroup $S$.

Then $a$ and $b$ are said to form an anomalous pair
if for all $n\in \mathbb{N}^+$, one of the following holds
\begin{align}
na < nb < (n+1)a \\
na > nb > (n+1)a
\end{align}
\end{definition}

\begin{theorem}
If $S$ is a non-Archimedean ordered semigroup, then there exists an anomalous pair.
\end{theorem}
\begin{proof}
Since $S$ is non-Archimedean, there exists $a,b\in S$ such that
$a$ and $b$ have the sign and are not mutually Archimedean.

First, we look at the case where $a$ and $b$ are positive.
Then since they are not mutually Archimedean, without loss of generality, for all $n\in \mathbb{N}^+$,
$na < b$.

Then we either have that $a + b \le b + a$ or that $b+a\le a + b$.
In the first case, we have that
\[
na + nb \le n(a+b) \le nb + na
\]
which means that, since $na < b$,
\[
nb < n(a+b) < (n+1)b
\]
And so $b$ and $a+b$ form an anomalous pair.

The other cases follow similarly.
\end{proof}

\begin{theorem}
An ordered semigroup without anomalous pairs is an Archimedean commutative semigroup.
\end{theorem}
\begin{proof}
Let $a$ and $b$ be elements of an ordered semigroup $S$.
If either $a$ or $b$ is zero, then they commute.

We begin with the case that $a$ and $b$ are positive.
If $a + b < b + a$, then for all $n\in \mathbb{N}^+$, we have that
\begin{align}
(n+1)(a+b) &= a + n(b+a) + b \\
&> n(b+a) + b \\
&> n(b+a) \\
&> n(a+b)
\end{align}
And so $a+b$ and $b+a$ form an anomalous pair.

The same for the case that $b + a < a + b$.
Therefore, we must have that $a+b = b+a$.

The case where $a$ and $b$ are negative follows similarly.

We now look at the case where $a$ is negative and $b$ is positive.
If the element $0$ exists and $a+b = 0$, then $a + b + a = a$ and so $b+a = 0$.
Therefore, $a$ and $b$ commute.

If $a + b$ is positive, then
\begin{align}
(a+b) + (a+b) &> a + b \\
(b + a) + b &> b \\
b + a &> 0
\end{align}

We already showed that positive elements commute and so
\[ (b+a) + b = b + (b + a)\]

Now we look at the case where $a+b < b+a$.
Then we have that
\begin{align}
2(a + b) &= a + ((b+a) + b) \\
&= a + (b + (b + a)) \\
&= (a + b) + (b + a) \\
&> (a + b) + (a + b) \\
&= 2(a + b)
\end{align}
Which is a contradiction.
We get a similar contradiction in the case that $b+a < a+b$.
Therefore, $a+b = b+a$.

The case where $a+b$ is negative follows similarly.
The case where $b$ is negative and $a$ is positive is symmetric.
\end{proof}