% In this file you should put the actual content of the blueprint.
% It will be used both by the web and the print version.
% It should *not* include the \begin{document}
%
% If you want to split the blueprint content into several files then
% the current file can be a simple sequence of \input. Otherwise It
% can start with a \section or \chapter for instance.

\section{Introduction}
We follow the paper ``On ordered semigroups'' by
N. G. Alimov.

\section{Content}

\begin{definition}\label{def:OrderedSemigroup}\lean{OrderedSemigroup}\leanok
An \textbf{ordered semigroup} $S$ is a semigroup with a partial order such that
for all $a,b,c\in S$ such that $a \le b$
\begin{enumerate}
    \item $a * c \le b * c$ and
    \item $c * a \le c * b$
\end{enumerate}
\end{definition}

\begin{definition}\label{def:OrderedCancelSemigroup}\lean{OrderedCancelSemigroup}\leanok
\uses{def:OrderedSemigroup}
An \textbf{ordered cancel semigroup} $S$ is an ordered semigroup such that for all $a,b,c\in S$,
if $a * b \le a * c$ then $b \le c$ and if $b * a \le c * a$ then $b \le c$.
\end{definition}

\begin{definition}\label{def:LinearOrderedSemigroup}\lean{LinearOrderedSemigroup}\leanok
\uses{def:OrderedSemigroup}
A \textbf{linear ordered semigroup} is an ordered semigroup where its partial order is a linear order.
\end{definition}

\begin{definition}\label{def:LinearOrderedCancelSemigroup}\lean{LinearOrderedCancelSemigroup}\leanok
\uses{def:OrderedCancelSemigroup}
A \textbf{linear ordered cancel semigroup} is an ordered cancel semigroup where its partial order is a linear order.
\end{definition}

\begin{definition}\label{def:positive}\lean{is_positive}\leanok
\uses{def:OrderedSemigroup}
An element $a$ of an ordered semigroup $S$ is \textbf{positive} if for all $x\in S$, $a*x > x$.
\end{definition}

\begin{definition}\label{def:negative}\lean{is_negative}\leanok
\uses{def:OrderedSemigroup}
An element $a$ of an ordered semigroup $S$
is \textbf{negative} if for all $x\in S$, $a*x < x$.
\end{definition}

\begin{definition}\label{def:one}\lean{is_one}\leanok
\uses{def:OrderedSemigroup}
An element $a$ of an ordered semigroup $S$
is \textbf{one} if for all $x\in S$, $a*x = x$.
\end{definition}

\begin{theorem}\label{thm:pos_neg_or_one}\lean{pos_neg_or_one}\leanok
\uses{def:positive, def:negative, def:one, def:LinearOrderedCancelSemigroup}
Each element $a$ of a linear ordered cancel semigroup $S$ is either positive, negative, or one.
\end{theorem}
\begin{proof}
Let $a\in S$ and $b\in S$. Since the $S$ is a linear order we have one of the following cases.

The first case is that $b * a > b$. Then for all $x\in S$ we have that $b * a * x > b * x$
and so $a * x > x$. Therefore, $a$ is positive.

The second case is that $b * a < b$. Then for all $x\in S$ we have that $b * a * x < b * x$
and so $a * x < x$. Therefore, $a$ is negative.

The last case is that $b * a = b$. Then for all $x\in S$ we have that $b * a * x = b * x$
and so $a * x = x$. Therefore, $a$ is zero.
\end{proof}

\begin{corollary}\label{lem:right_forall}\lean{pos_right_pos_forall, neg_right_neg_forall, one_right_one_forall}\leanok
\uses{def:positive, def:negative, def:one}
Let $a$ be an element of a linear ordered cancel semigroup $S$. 

If there exists a $b\in S$ such that $b * a > b$, $a$ is positive.

If there exists a $b\in S$ such that $b * a < b$, $a$ is negative.

If there exists a $b\in S$ such that $b * a = b$, $a$ is one.
\end{corollary}

\begin{theorem}\label{thm:neg_lt_pos}\lean{neg_lt_pos}\leanok
\uses{def:positive, def:negative}
Let $a$ and $b$ be elements of a linear ordered cancel semigroup $S$.

If $a$ is negative and $b$ is positive, then $a < b$.
\end{theorem}
\begin{proof}
Let $a$ be negative and $b$ be positive. Then for all $x\in S$ we have that $a * x < x$.
Likewise, for all $x\in S$ we have that $b * x > x$.
Therefore, $a * x < x < b + x$ and so $a < b$.
\end{proof}

We endow our ordered semigroup with scalar multiplication by the positive natural numbers.

\begin{lemma}\label{split_first_and_last}\lean{split_first_and_last_factor_of_product}\leanok
Let $a$ and $b$ be elements of a semigroup $S$.

For all $n > 1$, $(a*b)^n = a * (b+a)^(n-1) * b$.
\end{lemma}
\begin{proof}
Let $n=2$. Then $(a+b)^2 = a * (b*a)^(2-1) * b$.

Assume that the statement holds for $n$.
Then we have that $(a*b)^(n+1) = a * b * (a*b)^n = a * b * a * (b*a)^(n-1) * b = a * (b*a)^n * b$.
\end{proof}

\begin{lemma}\label{thm:comm_ineq}\lean{comm_factor_le, comm_swap_le, comm_dist_le}\leanok
Let $a$ and $b$ be elements of an ordered semigroup $S$.

If $a * b < b * a$, then for any $n\in \mathbb{N}^+$, we have that
\[a^n * b^n < (a*b)^n < (b*a)^n < b^n * a^n\]
\end{lemma}
\begin{proof}
If $n=1$, then we are done.

Assume that the statement holds for $n$.
Then we have that
\begin{align}
a^(n+1) + b^(n+1) &= a * a^n * b^n * b \\
\text{by the induction hypothesis}\\
&< a * (a * b)^n * b \\
\text{since $a*b < b*a$}\\
&< a * (b * a)^n * b \\
\text{by the previous lemma}\\
&= (a * b)^(n+1) \\
&< (b * a)^(n+1) \\
\text{by the previous lemma}\\
&= b * (a * b)^n * a \\
&< b * (b * a)^n * a \\
\text{by the induction hypothesis}\\
&< b * b^n * a^n * a \\
&= b^(n+1) * a^(n+1)
\end{align}
\end{proof}

\begin{definition}\label{def:arch_wrt}\lean{is_archimedean_wrt}\leanok
\uses{def:positive, def:negative}
Let $a$ and $b$ be elements of an ordered semigroup $S$ that are not one.

Then $a$ is said to be Archimedean with respect to $b$
if there exists an $N\in \mathbb{N}^+$ such that for $n > N$,
the inequality $b < a^n$ holds if $b$ is positive,
and the inequality $b > a^n$ holds if $b$ is negative.
\end{definition}

\begin{definition}\label{def:arch}\lean{is_archimedean}\leanok
\uses{def:one, def:arch_wrt}
An ordered semigroup is Archimedean if any two of its elements
of the same sign are mutually Archimedean.
\end{definition}

\begin{definition}\label{def:anomalous_pair}\lean{anomalous_pair}\leanok
Let $a$ and $b$ be elements of an ordered semigroup $S$.

Then $a$ and $b$ are said to form an anomalous pair
if for all $n\in \mathbb{N}^+$, one of the following holds
\begin{align}
a^n < b^n < a^(n+1) \\
a^n > b^n > a^(n+1)
\end{align}
\end{definition}

\begin{theorem}\label{non_arch_anomalous}\lean{non_archimedean_anomalous_pair}\leanok
\uses{def:anomalous_pair, def:arch}
If $S$ is a non-Archimedean linear ordered cancel semigroup, then there exists an anomalous pair.
\end{theorem}
\begin{proof}
Since $S$ is non-Archimedean, there exists $a,b\in S$ such that
$a$ and $b$ have the sign and are not mutually Archimedean.

First, we look at the case where $a$ and $b$ are positive.
Then since they are not mutually Archimedean, without loss of generality, for all $n\in \mathbb{N}^+$,
$a^n < b$.

Then we either have that $a * b \le b * a$ or that $b*a\le a * b$.
In the first case, we have that
\[
a^n + b^n \le (a*b)^n \le b^n + a^n
\]
which means that, since $a^n < b$,
\[
b^n < (a*b)^n < b^(n+1)
\]
And so $b$ and $a*b$ form an anomalous pair.

The other cases follow similarly.
\end{proof}

\begin{theorem}
A linear ordered cancel semigroup without anomalous pairs is an Archimedean commutative semigroup.
\end{theorem}
\begin{proof}
Let $a$ and $b$ be elements of an ordered semigroup $S$.
If either $a$ or $b$ is zero, then they commute.

We begin with the case that $a$ and $b$ are positive.
If $a + b < b + a$, then for all $n\in \mathbb{N}^+$, we have that
\begin{align}
(n+1)(a+b) &= a + n(b+a) + b \\
&> n(b+a) + b \\
&> n(b+a) \\
&> n(a+b)
\end{align}
And so $a+b$ and $b+a$ form an anomalous pair.

The same for the case that $b + a < a + b$.
Therefore, we must have that $a+b = b+a$.

The case where $a$ and $b$ are negative follows similarly.

We now look at the case where $a$ is negative and $b$ is positive.
If the element $0$ exists and $a+b = 0$, then $a + b + a = a$ and so $b+a = 0$.
Therefore, $a$ and $b$ commute.

If $a + b$ is positive, then
\begin{align}
(a+b) + (a+b) &> a + b \\
(b + a) + b &> b \\
b + a &> 0
\end{align}

We already showed that positive elements commute and so
\[ (b+a) + b = b + (b + a)\]

Now we look at the case where $a+b < b+a$.
Then we have that
\begin{align}
2(a + b) &= a + ((b+a) + b) \\
&= a + (b + (b + a)) \\
&= (a + b) + (b + a) \\
&> (a + b) + (a + b) \\
&= 2(a + b)
\end{align}
Which is a contradiction.
We get a similar contradiction in the case that $b+a < a+b$.
Therefore, $a+b = b+a$.

The case where $a+b$ is negative follows similarly.
The case where $b$ is negative and $a$ is positive is symmetric.
\end{proof}